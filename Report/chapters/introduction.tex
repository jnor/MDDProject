\pagestyle{headings}  % switches on printing of running heads

%\addtocmark{Hamiltonian Mechanics} % additional mark in the TOC

\title{Lalalalalallalal la lal ala la l}

\chapter*{Introduction}
Content Management Systems are used for creating and controlling content of web
applications. The concept of a Content Management System (CMS) reaches back to the
early 1970s to the era of Mainframe Content Managers.

With the advent of the internet, the requirements of Content Management Systems
have been constantly increasing, resulting in very complex systems. The increase in
complexity has led to the development of many web frameworks which all aim to assist
in the implementation of web applications.

We acknowledge that the architecture and specifics of Content Management Systems can
be hard to master. Domain experts, such as web designers, also spend increasing amount of
time and resources on customizing the Content Management Systems to meet customer demands.



\section*{State of the art}
Viðar Svansson and Roberto E. Lopez-Herrejon1, address the issue of increased complexity
of Content Management Systems and the needs for a domain-specific language for the
implementation of applications. The paper of Viðar Svansson and Roberto E. Lopez-Herrejon
also contains a description of a web specific language (WSL). The purpose of WSL is to increase
the productivity of developers by describing the business domain at a higher abstraction level
and automate the implementation process.

Eelco Visser has addressed the same issue of Content Management Systems and has developed a
domain-specific language, WebDSL2 , for the development of web applications. These papers
present generic domain-specific languages which are not aimed for any specific framework or
programming language.


\section*{Specific problem}
Content is information intended for a specific end-user audience, examples of
content are pictures and text documents.

Content-types3 are the heart of a Content Management System as they define the properties of the
content. For example a picture must contain a title, description and an image file. Many Content
Management Systems come with predefined content-types; however these are rarely enough to meet
the demands of the customer. To meet the demands, new content-types must be created and added to
the Content Management System.

The creation of customized content-types can be a very time-consuming process which involves
knowledge of the Content Management System architecture. Even Content Management Systems which
promote ‘easy to extend’ functionality, still require several steps of manual implementation of
the customized content-types.

The steps involve writing a description of the content-type, writing an editor for the content-type,
writing a HTML view of the content-type and finally registering the new content-type. This project
focuses on automating the steps required in the creation of new content-types, thus enabling the
domain experts to describe the content-type at a higher abstraction level. We will conduct an
architectural study and analysis of a chosen open source Content Management System Jease.

From this study we will raise the aspect of customization to a higher level of
abstraction by developing a domain-specific language which will facilitate creation of
customized content-types. This domain-specific language will assist in the process of creation of new
content-types by automating all of the required steps. We will also implement an editor for the
domain-specific language and transform the language into concrete code.


